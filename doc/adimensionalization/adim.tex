\documentclass[a4paper,12pt, draft]{article}

\usepackage[english]{babel}
\usepackage{amsmath}
\usepackage{mathtools}
\usepackage[%
    activate={true,nocompatibility},%
    final,%
    tracking=true,%
    kerning=true,%
    spacing=true,%
    factor=1100,%
    stretch=20,%
    shrink=20]{microtype}
\usepackage{nicefrac}
\usepackage{array}   % for \newcolumntype macro
\usepackage{tabularx}
\usepackage{framed}

\microtypecontext{spacing=nonfrench}
\setlength{\parindent}{0pt}


\newcolumntype{L}{>{$\displaystyle}X<{$}}


\newcommand{\Def}{\stackrel{\mathrm{def}}{=}}
\newcommand{\efield}{\ensuremath{\mathbf{E}}}
\newcommand{\displ}{\ensuremath{\mathbf{W}}}
\newcommand{\qtest}{\ensuremath{\mathbf{Q}}}
\newcommand{\ttest}{\ensuremath{\xi}}

\newcommand{\cmagn}{\ensuremath{K}}
\newcommand{\muref}{\ensuremath{\delta}}
\newcommand{\adim}[1]{\ensuremath{\underline{#1}}}


\begin{document}
Let us start with the equation that describes the electric potential:
\begin{equation*}
 \nabla\cdot\left( \varepsilon \nabla V\right) = q \left( c - n + p \right)
\end{equation*}

that we have rewritten as: find $V$, $E$ and $\widehat{V}$ such that
\begin{equation} \label{eq:poissonwithdim}
 \begin{cases}
  \left(\efield, \qtest \right) - \left(V,\nabla \cdot \qtest\right) +
    \left<\widehat{V}, \qtest \cdot \nu \right>     = 0 \\
  - \left(\varepsilon \efield, \nabla z\right) - q \left(c -n + p,z\right) +
    \left<\varepsilon \efield \cdot \nu, z\right> + \tau_{V} \big<V - \widehat{V}, z \big> = 0 \\
  \left<\varepsilon \efield \cdot \nu, \ttest \right> + \tau_{V} \big<V - \widehat{V}, \ttest \big> 
=0
 \end{cases}
\end{equation}

We define a value $K$ that represent the magnitude of the doping
\begin{equation*}
  \cmagn \Def \operatorname{supess}_\Omega \left| c \right|
\end{equation*}
and, therefore, we define the following dimensionless quantities
\begin{align*}
  \adim{c} & \Def \frac{c}{\cmagn} \\
  \adim{n} & \Def \frac{n}{\cmagn} \\
  \adim{p} & \Def \frac{p}{\cmagn}
\end{align*}

Moreover, we define the quantity $T_{V}$ as the maximum value reached by the thermal voltage
\[
 T_{V} \Def \frac{\mathrm{k}_{\mathrm{B}}}{q} \operatorname{sup}_{\Omega}T
\]
and the diameter of the domain $L$ so that we can define the dimensionless quantities
\begin{align*}
 \adim{V} & \Def \frac{V}{T_{V}} \\
 \adim{\efield} & \Def \frac{L}{T_{V}}\efield \\
 \adim{\widehat{V}} & \Def \frac{\widehat{V}}{T_{V}} \\
\end{align*}
We define also
\begin{equation}\label{eq:adimx} \adim{x} \Def \frac{x}{L} \end{equation}

In the following equations, we will indicate the gradient with the symbol $\nabla_{x}$ to specify 
that the derivation involves the spacial variable $x$. Instead, we will use the symbol 
$\nabla_{\adim{x}}$ if the derivation involves the dimensionless variable $\adim{x}$. Because of 
the equation~\ref{eq:adimx}, we have that
\[ \nabla_{\adim{x}}f = L \nabla_{x} f \]
Finally, we must also define
\[ \adim{\nu} \Def L \nu \]
so that also the normal vector is independent from the spacial dimension of our problem.

With the previous definitions, the first equation of system~\ref{eq:poissonwithdim} becomes
\[
  \frac{T_{V}}{L} \left(\adim{\efield}, \qtest \right) 
    - \frac{T_{V}}{L} \left(\adim{V},\nabla_{\adim{x}} \cdot \qtest\right) 
    + \frac{T_{V}}{L}\left<\adim{\widehat{V}}, \qtest \cdot \adim{\nu} \right>  = 0
\]
that can be simplified as
\[
  \left(\adim{\efield}, \qtest \right) - \left(\adim{V},\nabla_{\adim{x}} \cdot \qtest\right) 
    + \left<\adim{\widehat{V}}, \qtest \cdot \adim{\nu} \right>  = 0
\]
For what concerns the second equation, instead, we obtain
\[
 - \frac{T_{V}}{L^2} \left(
     \varepsilon \adim{\efield}, \nabla_{\adim{x}} z
   \right) - q \cmagn \left(\adim{c} -\adim{n} + \adim{p}, z\right) +
   \frac{T_{V}}{L^2}  \left<\varepsilon \adim{\efield} \cdot \adim{\nu}, z\right> +
   T_{V} \tau_{V} \big<\adim{V} - \adim{\widehat{V}}, z \big> = 0
\]
If we divide by $q \cmagn$, 
we obtain
\[
 - \frac{T_{V}}{q \cmagn L^2} \left(
     \varepsilon \adim{\efield}, \nabla_{\adim{x}} z
   \right) - \left(\adim{c} \! - \! \adim{n} \! + \! \adim{p}, z\right) +
   \frac{T_{V}}{q \cmagn L^2}  \left<\varepsilon \adim{\efield} \cdot \adim{\nu}, z\right> +
   \frac{T_{V}}{q \cmagn} \tau_{V} \big<\adim{V} - \adim{\widehat{V}}, z \big> = 0
\]
Defining
\begin{align*}
 \adim{\varepsilon} & \Def \frac{T_{V}}{q \cmagn L^2} \varepsilon \\
 \adim{\tau_{V}} & \Def \frac{T_{V}}{q \cmagn} \tau_{V}
\end{align*}
the equation becomes
\[
 - \left(\adim{\varepsilon} \adim{\efield}, \nabla_{\adim{x}} z \right) 
   - \left(\adim{c} - \adim{n} + \adim{p}, z\right)
   + \left<\adim{\varepsilon} \adim{\efield} \cdot \adim{\nu}, z\right>
   + \adim{\tau_{V}} \big<\adim{V} - \adim{\widehat{V}}, z \big> = 0
\]

Finally, using the same substitution, the third equation of system~\ref{eq:poissonwithdim} becomes
\[ 
  \frac{T_{V}}{L^2} \left<\adim{\varepsilon} \adim{\efield} \cdot \adim{\nu}, \ttest \right> 
    + \frac{T_{V}}{L^2} \adim{\tau_{V}} \big<\adim{V} - \adim{\widehat{V}}, \ttest \big> = 0
\]
which can be of course divided by $\frac{T_{V}}{L^2}$.

Now we are able to rewrite the system~\ref{eq:poissonwithdim} using only dimensionless quantities, 
and we obtain
\[ \label{eq:poissonadim}
  \begin{cases}
  \left(\adim{\efield}, \qtest \right) - \left(\adim{V},\nabla_{\adim{x}} \cdot \qtest\right) +
    \left<\adim{\widehat{V}}, \qtest \cdot \adim{\nu} \right>     = 0 \\
  - \left(\adim{\varepsilon} \adim{\efield}, \nabla_{\adim{x}} z\right)
    - \left(\adim{c} - \adim{n} + \adim{p}, z\right)
    + \left<\adim{\varepsilon} \adim{\efield} \cdot \adim{\nu}, z\right>
    + \adim{\tau_{V}} \big<\adim{V} - \adim{\widehat{V}}, z \big> = 0 \\
  \left<\adim{\varepsilon} \adim{\efield} \cdot \adim{\nu}, \ttest \right>
    + \adim{\tau_{V}} \big<\adim{V} - \adim{\widehat{V}}, \ttest \big> =0
 \end{cases}
\]

Let us now take into account the following equation, that imposes the conservation of the charge
\[ \nabla \cdot \left( \mu \efield n + D \nabla n \right) = R \]
Like we did in the previous case, we rewrite the equation in a form that is suitable for the HDG 
methods:
\begin{equation} \label{eq:chargeconswithdim}
 \begin{cases}
  \left(\displ, \qtest \right) - \left(n,\nabla \cdot \qtest\right) +
    \left<\widehat{n}, \qtest \cdot \nu \right> = 0 \\
  - \! \left(\mu \efield n,\! \nabla z\right) \! + \! \left(D \displ, \nabla z \right)
    \! + \! \left<\mu \efield n \! \cdot \! \nu, z\right>
    \! - \! \left<D \displ \! \cdot \! \nu, z \right> 
    \! - \! \tau_{n} \big<n \! - \! \widehat{n}, z \big> \! = \! (R, z) \\
  \left<\mu \efield n \cdot \nu, \ttest \right> - \left<D \displ \cdot \nu, \ttest \right> 
    - \tau_{n} \big<n - \widehat{n}, \ttest \big> = 0
 \end{cases}
\end{equation}

The first equation of this system is totally analogous to the first equation of the
system~\ref{eq:poissonwithdim}. Therefore, we proceed in the same way: we define
\[ \adim{\displ} \Def \frac{L}{\cmagn}\displ \]
and we multiply the first equation by $\frac{\cmagn}{L}$ to obtain
\[
  \left(\adim{\displ}, \qtest \right) - \left(\adim{n},\nabla_{\adim{x}} \cdot \qtest\right) +
    \left<\adim{\widehat{n}}, \qtest \cdot \adim{\nu} \right> = 0
\]

For the second equation, we define a reference value $\muref$ for the electron mobility 
(and we will use the same value also for the hole mobility) to obtain

\[ \adim{\mu} \Def \frac{\mu}{\muref} \]

Therefore, our previous equation becomes
\[
  \begin{multlined}
  -  \frac{\muref T_{V} \cmagn}{L^2} \left(
      \adim{\mu} \adim{\efield} \adim{n}, \nabla_{\adim{x}} z
    \right)
    +  \frac{\cmagn}{L^2}\left(D \adim{\displ}, \nabla_{\adim{x}} z \right)
    + \frac{\muref T_{V} \cmagn}{L^2}\left<
        \adim{\mu} \adim{\efield} \adim{n} \cdot \adim{\nu}, z
      \right> \\
    - \frac{\cmagn}{L^2}\left<D \adim{\displ} \cdot \adim{\nu}, z \right> 
    - \cmagn \tau_{n} \big<\adim{n} - \adim{\widehat{n}}, z \big> = \left( R, z \right)
  \end{multlined}
\]

If we multiply it by the inverse of the coefficient
\[ \frac{\muref T_{V} \cmagn}{L^2} \]
we get
\[
  \begin{multlined}
  -  \left(
      \adim{\mu} \adim{\efield} \adim{n}, \nabla_{\adim{x}} z
    \right)
    +  \frac{1}{\muref T_{V}}\left(D \adim{\displ}, \nabla_{\adim{x}} z \right)
    + \left<
        \adim{\mu} \adim{\efield} \adim{n} \cdot \adim{\nu}, z
      \right> \\
    - \frac{1}{\muref T_{V}}\left<D \adim{\displ} \cdot \adim{\nu}, z \right> 
    - \frac{L^2}{\muref T_{V}} \tau_{n} \big<\adim{n} - \adim{\widehat{n}}, z \big>
    = \frac{L^2}{\muref T_{V} \cmagn} \left( R, z \right)
  \end{multlined}
\]

We furthermore define
\begin{align*}
  \adim{D} & \Def \frac{D}{\muref T_{V}} \\
  \adim{\tau_n} & \Def \frac{L^2}{\muref T_{V}} \tau_n \\
  \adim{R} & \Def \frac{L^2}{\muref T_{V} \cmagn} R
\end{align*}
we can rewrite the previous equation in a simpler form as
\[
  \begin{multlined}
  -  \left(
      \adim{\mu} \adim{\efield} \adim{n}, \nabla_{\adim{x}} z
    \right)
    +  \left(\adim{D} \adim{\displ}, \nabla_{\adim{x}} z \right)
    + \left<
        \adim{\mu} \adim{\efield} \adim{n} \cdot \adim{\nu}, z
      \right> \\[5pt]
    - \left< \adim{D} \adim{\displ} \cdot \adim{\nu}, z \right> 
    - \adim{\tau_{n}} \big<\adim{n} - \adim{\widehat{n}}, z \big>
    = \left( \adim{R}, z \right)
  \end{multlined}
\]

Finally, let us consider the third equation of system~\ref{eq:chargeconswithdim}. In this case, we 
have
\[
  \frac{\muref T_{V} \cmagn}{L^2}\left<\adim{\mu}\adim{\efield}\adim{n}\cdot\adim{\nu},\ttest\right>
    - \frac{\muref T_{V} \cmagn}{L^2}\left<\adim{D} \adim{\displ} \cdot \adim{\nu}, \ttest \right> 
    - \frac{\muref T_{V} \cmagn}{L^2} \adim{\tau_{n}} \big<
        \adim{n} - \adim{\widehat{n}}, \ttest 
       \big> = 0
\]
which, of course, can be simplified multiplying by
\[ \frac{L^2}{\muref T_{V} \cmagn} \]

Like we did before for the system~\ref{eq:poissonwithdim}, we can now rewrite the 
system~\ref{eq:chargeconswithdim} using only dimensionless quantities:

\[ \label{eq:chargeconadim}
  \begin{dcases}
  \left(\adim{\displ}, \qtest \right) - \left(\adim{n},\nabla_{\adim{x}} \cdot \qtest\right)
    + \left<\adim{\widehat{n}}, \qtest \cdot \adim{\nu} \right> = 0 \\[10pt]
  \begin{multlined}[b][11cm]
  -  \left(
      \adim{\mu} \adim{\efield} \adim{n}, \nabla_{\adim{x}} z
    \right)
    +  \left(\adim{D} \adim{\displ}, \nabla_{\adim{x}} z \right)
    + \left<
        \adim{\mu} \adim{\efield} \adim{n} \cdot \adim{\nu}, z
      \right> \\
    - \left< \adim{D} \adim{\displ} \cdot \adim{\nu}, z \right> 
    - \adim{\tau_{n}} \big<\adim{n} - \adim{\widehat{n}}, z \big>
    = \left( \adim{R}, z \right)
  \end{multlined}\\[10pt]
  \left< \adim{\mu} \adim{\efield} \adim{n} \cdot \adim{\nu}, \ttest \right>
    - \left< \adim{D} \adim{\displ} \cdot \adim{\nu}, \ttest \right> 
    - \adim{\tau_{n}} \big<\adim{n} - \adim{\widehat{n}}, \ttest \big> = 0
 \end{dcases}
\]

As a final remark, we want to report all the definitions of the dimensionless quantities in a table.


\begin{framed}
\begin{center}
\begin{tabularx}{.8\textwidth}{L L}
\cmagn \Def \operatorname{supess}_\Omega \left| c \right| &  
  T_{V} \Def \frac{\mathrm{k}_{\mathrm{B}}}{q} \operatorname{sup}_{\Omega}T
\end{tabularx}
\end{center}
\end{framed}

\setlength{\extrarowheight}{20pt}
\begin{framed}
\begin{center}
\begin{tabularx}{\textwidth}{L L L}
 \adim{V} \Def \frac{V}{T_{V}}  
   & \adim{n} \Def \frac{n}{\cmagn}
   & \adim{p} \Def \frac{p}{\cmagn} \\
 \adim{\efield} \Def \frac{L}{T_{V}}\efield
   & \adim{\displ_{n}} \Def \frac{L}{\cmagn} \displ_{n}
   & \adim{\displ_{p}} \Def \frac{L}{\cmagn} \displ_{p} \\
 \adim{\widehat{V}} \Def \frac{\widehat{V}}{T_{V}}
   & \adim{\widehat{n}} \Def \frac{\widehat{n}}{\cmagn}
   & \adim{\widehat{p}} \Def \frac{\widehat{p}}{\cmagn}\\
 \adim{\tau_{V}} \Def \frac{T_{V}}{q \cmagn} \tau_{V}
   & \adim{\tau_n} \Def \frac{L^2}{\muref T_{V}} \tau_n
   & \adim{\tau_p} \Def \frac{L^2}{\muref T_{V}} \tau_p\\
 \adim{\varepsilon} \Def \frac{T_{V}}{q \cmagn L^2} \varepsilon
   &  \adim{\mu_n} \Def \frac{\mu_n}{\muref}
   &  \adim{\mu_p} \Def \frac{\mu_p}{\muref} \\
 \adim{R} \Def \frac{L^2}{\muref T_{V} \cmagn} R
   & \adim{D_n} \Def \frac{D_n}{\muref T_{V}}
   & \adim{D_p} \Def \frac{D_p}{\muref T_{V}}
\end{tabularx}
\end{center}
\end{framed}

\end{document}
