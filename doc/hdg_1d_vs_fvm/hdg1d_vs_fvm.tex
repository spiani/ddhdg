\documentclass[a4paper,11pt, draft]{article}

\usepackage{amssymb}
\usepackage{amsmath}
\usepackage{amsthm}

\usepackage{mathtools}

%opening
\title{A paper}
\author{Some authors}

\setlength{\marginparwidth}{2.5cm}
\usepackage{todonotes}


\newtheorem{prop}{Proposition}
\newtheorem{lemma}{Lemma}[prop]

\newcommand{\diver}[1]{\ensuremath{\nabla \cdot #1}}
\newcommand{\closure}[1]{\ensuremath{\overline{#1}}}
\newcommand{\interior}[1]{\ensuremath{{\kern0pt#1}^{\mathrm{o}}}}
\newcommand{\Lspace}[1]{\ensuremath{L^{#1}}}
\newcommand{\Def}{\stackrel{\mathrm{def}}{=}}
\newcommand{\dualt}{\ensuremath{\mathcal{T}^d}}

% Enable \reallywidehat command
\usepackage{scalerel,stackengine}
\stackMath
\newcommand\reallywidehat[1]{%
\savestack{\tmpbox}{\stretchto{%
  \scaleto{%
    \scalerel*[\widthof{\ensuremath{#1}}]{\kern-.6pt\bigwedge\kern-.6pt}%
    {\rule[-\textheight/2]{1ex}{\textheight}}%WIDTH-LIMITED BIG WEDGE
  }{\textheight}%
}{0.5ex}}%
\stackon[1pt]{#1}{\tmpbox}%
}
\parskip 1ex
%

\begin{document}

\section{Description of the problem}


Let $\Omega \subset \mathbb{R}^n$ be a open bounded Lipschitz domain. Let $E \in 
[L^{\infty}(\Omega)]^n$ and $D \in L^{\infty}(\Omega)$ such that exists a constant
$D_0 < 0 $ such that for every $x \in \Omega$, $D(x) \leq D_0$. Let also $f \in H^1(\Omega)$.

We want to solve the following problem: find a function $n$ such that
\[
 \begin{cases}
  \nabla \cdot \left(n E + D \nabla n \right) = f & \textrm{on } \Omega\\
  n = n_0 & \textrm{on } \partial \Omega
 \end{cases}
\]


\section{HDG method}
We propose two different weak formulations of the previous problem. In the first one, we define
\[ W \Def - \nabla n \]
and, therefore, the previous problem becomes
\begin{equation} \label{eq:problem1_strong}
 \begin{cases}
  W + \nabla n = 0 & \textrm{on } \Omega \\
  \nabla \cdot \left(n E - D W \right) = f & \textrm{on } \Omega\\
  n = n_0 & \textrm{on } \partial \Omega
 \end{cases}
\end{equation}

In the second one, instead, we define
\[ J \Def n E + D \nabla n \]
obtaining the following problem
\begin{equation}\label{eq:problem2_strong}
 \begin{cases}
  J - n E - D \nabla n = 0 & \textrm{on } \Omega \\
  \nabla \cdot J = f & \textrm{on } \Omega\\
  n = n_0 & \textrm{on } \partial \Omega
 \end{cases}
\end{equation}

The problem~\ref{eq:problem1_strong} can be rewritten in a weak form as: find a function $n \in
H^1(\Omega)$ and a function $W \in [H^1(\Omega)]^n$ such that for every function $z \in 
H^1_0(\Omega)$ and for every function $Q \in [H^1(\Omega)]^n$
\begin{equation} \label{eq:problem1_weak}
 \begin{cases}
  (W, Q) - (n, \nabla \cdot Q) + \left<n, Q \cdot \nu \right>_{\partial \Omega} = 0 \\
  - (n E, \nabla z) + \left<n E \cdot \nu, z \right> + (D W, \nabla z) - \left<D W \cdot \nu, 
z\right> = (f, z) \\
  n|_{\partial \Omega} = n_0 |_{\partial \Omega}
 \end{cases}
\end{equation}

In the very same way, the problem~\ref{eq:problem2_strong} can be rewritten as: find a function $n 
\in H^1(\Omega)$ and a function $J \in [H^1(\Omega)]^n$ such that for every function \mbox{$z \in 
H^1_0(\Omega)$} and for every function $Q \in [H^1(\Omega)]^n$
\begin{equation} \label{eq:problem2_weak}
\begin{cases}
 (J, Q) - (n E, Q) + (n, \nabla \cdot (D Q)) - \left<n, DQ \cdot \nu \right> = 0 \\
 - (J, \nabla z) + \left<J \cdot \nu, z \right> = (f, z) \\
   n|_{\partial \Omega} = n_0 |_{\partial \Omega}
\end{cases}
\end{equation}


Let $\mathcal{T}$ be a triangulation of $\Omega$ and let $\Xi$ be the skeleton of the triangulation 
$\mathcal{T}$, obtained by the union of the boundary of the cells of $\mathcal{T}$. We will also 
define the set $\Xi_0$ as
\[ \Xi_0 \Def \Xi \setminus \partial \Omega \]
In order of 
apply the HDG method to the previous problems, we define an auxiliary function 
\[
  \widehat{n} \colon \Xi \longrightarrow \mathbb{R}
\]
and we apply the weak formulation on the broken space obtained from the disjoint union of the cells.
Therefore, for the first formulation, we get the following problem: find $n \in H^1(\mathcal{T})$, 
$W \in [H^1(\mathcal{T})]^n$ and $\widehat{n} \in L^2(\Xi)$ such that for every function $z \in 
H^1(\mathcal{T})$, $Q \in [H^1(\mathcal{T})]^n$ and $\mu \in L^2(\Xi_0)$
\begin{equation}\label{eq:problem1_hdg}
 \begin{cases}
   (W, Q) - (n, \nabla \cdot Q) + \left<\widehat{n}, Q \cdot \nu \right> = 0 \\
  - (n E, \nabla z) + (D W, \nabla z) + \left<(n E - D W) \cdot \nu, z\right>^{\wedge} = (f, z) \\
  \left<(n E - D W) \cdot \nu, \mu \right>^{\wedge} = 0 \\
  \widehat{n}|_{\partial \Omega} = n_0 |_{\partial \Omega}
 \end{cases}
\end{equation}
where the third equation enforces the (weak) continuity of the normal fluxes. The term
\[ \left<(n E - D W) \cdot \nu, z\right>^{\wedge} \]
represents the numerical flux and is defined as
\[ \left<(n E - D W) \cdot \nu, z\right>^{\wedge} \Def \left<nE \cdot \nu , z\right> - \left< 
D W \cdot \nu, z\right> - \tau \left< n - \widehat{n}, z\right> \]

For what concerns the formulation~\ref{eq:problem2_weak}, with the same technique we obtain that 
the system of equations we need to solve are
\begin{equation}\label{eq:problem2_hdg}
\begin{cases}
 (J, Q) - (n E, Q) + (n, \nabla \cdot (D Q)) - \left<\widehat{n}, DQ \cdot \nu \right> = 0 \\
 - (J, \nabla z) + \left<J \cdot \nu, z \right>^{\wedge} = (f, z) \\
   \left<J \cdot \nu, \mu \right>^{\wedge} = 0 \\
   \widehat{n}|_{\partial \Omega} = n_0 |_{\partial \Omega}
\end{cases} 
\end{equation}
where we define
\[
  \left<J \cdot \nu, z \right>^{\wedge} \Def \left<J \cdot \nu, z \right> + \tau \left<n - 
\widehat{n}, z\right>
\]

\section{A simplified case}
In this section, we will assume the hypothesis that $n = 1$ and that, therefore, $\Omega$ is a 
segment of $\mathbb{R}$.
$T_i$ will be the $i$--th element of the triangulation $\mathcal{T}$ of $\Omega$. The length of 
the element $T_i$ will be indicated with $h_i$.

Finally, We will also suppose that $D$, $E$ and $f$ are constant functions.


\begin{prop}
If we approximate the space of the functions of the problem~\ref{eq:problem1_hdg} and of the 
problem~\ref{eq:problem2_hdg} with the space of the functions that are constant on each element 
(i.e. we use polynomials of degree 0), the two formulations are equivalent.
\end{prop}

\begin{proof}
 Let us suppose that $\mathcal{T}$ is made of $n$ elements, $T_1, \ldots, T_n$. We will use $n_k$ 
to define the value that the unknown function $n$ assumes on the element $T_k$, and, in the same 
way, we also define $W_k$ and $J_k$.

Moreover, we will use $F_k$ to refer to the face (the point, in this case) on the left of the 
element $T_i$, while $F_{k+1}$ is the face to the right. Finally, $\widehat{n}_k$ is the value of 
the function $\widehat{n}$ on the face $F_k$.

Therefore, the first equation of system~\ref{eq:problem1_hdg} becomes
\[ h_k W_k - \widehat{n}_k + \widehat{n}_{k + 1} = 0 \]
for every $k$ in $\{1, \ldots, n\}$. For the second equation, all the terms on the left hand side 
disappear beside the one that are multiplied by $\tau$ (because the gradient of a constant is 0 and 
because of the fact that multiply by the normal vector changes the signs). Then we have
\[ - \tau (2 n_k - \widehat{n}_{k+1} - \widehat{n}_k) = h_k f \]
Finally, the last equation becomes
\[
  E n_{k} - E n_{k - 1} - D W_{k} + D W_{k - 1} - \tau (n_k + n_{k - 1} - 2\widehat{n}_k) = 0
\]
for every $k$ in $\{2, \ldots, n \}$, while $\widehat{n}_1 = n_0(0)$ and $\widehat{n}_{k + 1} = 
n_0(1)$ because of the boundary conditions.

Putting everything together, we obtain
\begin{equation}\label{eq:problem1_system}
 \begin{cases}
  \displaystyle h_k W_k - \widehat{n}_k + \widehat{n}_{k + 1} = 0  & \forall k \in \{1, 
\ldots, n\} \\
  \displaystyle -\tau (2 n_k - \widehat{n}_{k+1} - \widehat{n}_k) = h_k f & \forall k \in 
\{1, \ldots, n\} \\
  E (n_{k} \! - \! n_{k - 1}) - D (W_{k} \! - \! W_{k - 1}) - \tau (n_k \! +\! n_{k - 1} \! - \! 
2\widehat{n}_k) = 0 & \forall k \in \{2, \ldots, n\} \\
 \widehat{n}_1 = n_0(0) \\
 \widehat{n}_{k+1} = n_0(1)
 \end{cases}
\end{equation}

In the very same way, we can do the same for the the equations of the system~\ref{eq:problem2_hdg}, 
obtaining
\begin{equation}\label{eq:problem2_system}
 \begin{cases}
  h_k J_k - E h_k n_k - D \widehat{n}_{k + 1} + D \widehat{n}_{k} = 0 & \forall k \in \{1,\ldots, 
n\} \\
  \tau (2 n_k - \widehat{n}_{k + 1} - \widehat{n}_{k}) = h_k f & \forall k \in \{1,\ldots, n\} \\
  J_k - J_{k - 1} + \tau (n_k  + n_{k - 1} - 2\widehat{n}_k ) = 0 & \forall k \in \{2,\ldots, n\} \\
 \widehat{n}_1 = n_0(0) \\
 \widehat{n}_{k+1} = n_0(1)
 \end{cases}
\end{equation}

As a first step we want to prove that the system~\ref{eq:problem1_system} and 
\ref{eq:problem2_system} have the same solution for what concerns the value of $\widehat{n}$. To to 
that, it is enough to replace the values of the first and second equation into the third one, 
obtaining for the system~\ref{eq:problem1_system}

\begin{equation}\label{eq:hdg_condensed_system}
\begin{multlined}[b][.80\textwidth]
 \left(\frac{E \! - \! \tau}{2} - \frac{D}{h_{k - 1}}\right)\widehat{n}_{k - 1} + \left(\tau + 
\frac{D}{h_{k}} + \frac{D}{h_{k - 1}}\right) \widehat{n}_{k} + \left(\frac{-\!E \! - \! 
\tau}{2} - \frac{D}{h_{k}} \right) \widehat{n}_{k + 1} = \\[10pt]
 \frac{Ef}{2\tau}(h_{k} - h_{k - 1}) - \frac{h_{k} + h_{k-1}}{2}f
\end{multlined}
\end{equation}

If we perform the analogous operation for the system\ref{eq:problem2_system}, we get the same 
equation (beside the sign) and, therefore, the values obtained for $\widehat{n}$ are the same. 
\todo{For some strange reason, I get the opposite sign for the right hand side, but this can not
be true. I need to double check this!}

Finally, the values for $n$ are also the same, because the second equation of 
system~\ref{eq:problem1_system} completely defines the values of $n_k$ once $\widehat{n}_{j}$ is 
known for every $j$.
\end{proof}

\section{The finite volume method}
Another possibility to solve our problem is given by the standard volume method. In this case, we 
consider a vertex centered grid so that the centers of the cells of this new triangulation 
\dualt are the faces of the triangulation $\mathcal{T}$.

This also means that the distance between the centers of the cell $S_k$ of \dualt and of the cell 
$S_{k+1}$ is $h_{k}$.

We will call $u_k$ the approximation values of the function $n$ that the 
finite volume method provides for each cell $S_k$ (to avoid confusion with $n_k$, the values 
generated by the HDG methods on the cell $T_k$).

In the classical version of the method, we approximate the flux between two cells (let's say 
from $S_k$ to $S_{k+1}$) with
\[ \mathcal{F}_{k} \Def  D \frac{u_{k + 1} - u_k}{h_k} + E \left( \frac{1}{2} u_{k+1} + \frac{1}{2} 
u_k \right) \]
Of course, if you want to consider the flux from $S_{k+1}$ to $S_k$, you will need to change the 
sign.

This generates a system of equations like the following
\begin{equation}\label{eq:standard_fvm}
  \left(\frac{D}{h_{k - 1}} - \frac{E}{2} \right) {u_{k - 1}} 
    - {\left(\frac{D}{h_{k - 1}} + \frac{D}{h_{k}}\right)} u_{k}
    + \left(\frac{D}{h_{k}} + \frac{E}{2} \right) u_{k + 1}
    = \frac{h_{k - 1} + h_{k}}{2} f 
\end{equation}

\begin{prop} \label{hdg_vs_fvm}
 If the elements of the triangulation $\mathcal{T}$ have the same length, then the values that 
the HDG method returns if $\tau = 0$ for the $\widehat{n}$ function on the faces of $\mathcal{T}$ 
are the same that the finite volume method returns on the cells of \dualt.
\end{prop}

\begin{proof}
 Let us suppose that 
 \[ h_1 = h_2 = \cdots = h_k = h \]
 Then the equations of the system of the finite volume methods become
 \[
  \left(\frac{D}{h} - \frac{E}{2} \right) {u_{k - 1}} 
    - \frac{2D}{h} u_{k}
    + \left(\frac{D}{h} + \frac{E}{2} \right) u_{k + 1}
    = h f 
 \]
 In the same way, if we take equation~\ref{eq:hdg_condensed_system} and we perform substitute $h_j$ 
with $h$ (for every $j$) and $\tau$ with 0, we get the same equation (with an opposite sign); this 
concludes the proof.
\end{proof}

\section{The Scharfetter--Gummel scheme}
The Scharfetter--Gummel scheme offers a way to stabilize the previous finite volume method. Indeed, 
it modifies the fluxes so that
\[ \widetilde{\mathcal{F}_{k}} \Def \frac{D}{h_k} \left(\! B\left(\frac{E h_k}{D}\right) u_{k+1} - 
B\left(-\frac{E h_k}{D}\right) u_k \right)  \]
where $B(x)$ is the Bernulli function defined as
\[ B(x) \Def \frac{x}{\mathrm{e}^x - 1} \]
Applying this flux, we get a system made of the following equations
%\[
%\begin{multlined}[b]
%  \frac{E}{\mathrm{e}^{\left(\frac{E h_{k - 1}}{D}\right)} \! - \! 1} u_{k - 1}
% \! + \! \! \left( \frac{E}{\mathrm{e}^{\left(\frac{- E h_{k - 1}}{D}\right)} \! -\! 1}
%    \! - \! \frac{E}{\mathrm{e}^{\left(\frac{E h_{k}}{D}\right)}\! -\! 1} \right)\! u_k
% \! - \! \frac{E}{\mathrm{e}^{\left(\frac{- E h_{k}}{D}\right)} \!-\! 1} u_{k + 1} = \\
% \frac{h_{k - 1} + h_{k}}{2} f 
%\end{multlined}
%\]
%
\begin{equation}\label{eq:sg_system}
\begin{multlined}[.98\displaywidth]
 \frac{D}{h_{k - 1}} B\!\left(\! \frac{E h_{k-1}}{D} \!\right) u_{k-1} - \left[ \! \frac{D}{h_{k - 
1}} B\!\left(\! \frac{- E h_{k-1}}{D} \!\right) + \frac{D}{h_k} 
B\!\left(\! \frac{E h_k}{D} \!\right) \! \right] u_k \\[10pt] + \frac{D}{h_k} B\!\left(\! \frac{- E 
h_k}{D} \!\right) u_{k + 1} = \frac{h_{k - 1} + h_{k}}{2} f
\end{multlined}
\end{equation}


Before going on, we want to show two basic properties of the Bernulli function that we will use 
later

\begin{lemma} \label{bernulli_inverse}
 For every $x \in \mathbb{R}$,
 \[ B(x) - B(-x) = x \]
\end{lemma}
\begin{proof}
 Indeed
 \[
  \begin{split}
   B(x) - B(-x) & = \frac{x}{\mathrm{e}^x - 1} - \frac{x}{1 - \mathrm{e}^{-x}} \\
                & = \frac{x - x \mathrm{e}^{x}}{\mathrm{e}^x - 1} \\
                & = -x
  \end{split}
 \]
\end{proof}

.
\begin{lemma}\label{bernulli_even}
 The function
 \[ f(x) \Def 2 B(x) + x \]
 is an even function.
\end{lemma}

\begin{proof}
Applying the lemma~\ref{bernulli_inverse}, we have
\[ f(x) = 2B(x) - B(x) + B(-x) = B(x) + B(-x)\]
and now the thesis is trivial.
\end{proof}

As we have seen in the proposition~\ref{hdg_vs_fvm}, choosing wisely the value of $\tau$, we can 
obtain this method using the HDG formulation (at least, in 1D and for uniform grids). 
\begin{prop}\label{hdg_vs_sg}
If the elements of the triangulation $\mathcal{T}$ have the same length $h$, then there exists 
a particular value of $\tau$
\[ \tau_0 \Def \frac{D}{h}\left[2B\left(\!\frac{Eh}{D}\!\right) + \frac{Eh}{D} - 2 \right] \]
such that the values that the HDG method returns if $\tau = \tau_0$ for the 
$\widehat{n}$ function on the faces of $\mathcal{T}$ are the same that the finite volume method 
 with the Scharfetter--Gummel scheme returns on the cells of \dualt.

\end{prop}

\begin{proof}
 Let us consider the equation~\ref{eq:hdg_condensed_system} and, in particular, the coefficient of 
$\widehat{n}_{k-1}$ that, when $h_j = h$ for every $j$, becomes
\[ \frac{E \! - \! \tau}{2} - \frac{D}{h} \]
If we replace $\tau$ with $\tau_0$, we obtain
\[ \frac{E}{2} - \frac{D}{h} B\left(\! \frac{Eh}{D}\!\right) + \frac{E}{2} + \frac{D}{h} - 
\frac{D}{h} = - \frac{D}{h} B\left(\! \frac{Eh}{D}\!\right) \]
which is, beside the sign, the coefficient of $u_{k-1}$ in the equation~\ref{eq:sg_system}.
In the very same way, the coefficient of $\widehat{n}_k$
\[ - \left( \tau + \frac{2D}{h} \right)\]
becomes
\begin{equation}\label{eq:hdg_coeff} \frac{D}{h}\left[2B\left(\!\frac{Eh}{D}\!\right) + 
\frac{Eh}{D} \right] \end{equation}
The coefficient of $u_k$ in the equation~\ref{eq:sg_system} instead is
\[ - \frac{D}{h}\left[B\left(\!\frac{-Eh}{D} \!\right) + B\left(\!\frac{Eh}{D} \!\right) \right]\]
which is equal to the coefficient~\ref{eq:hdg_coeff} (beside the sign) because of the 
lemma~\ref{bernulli_inverse}.

Finally, the last coefficient of the left hand side of the equation~\ref{eq:hdg_condensed_system} is
\[ \left(\frac{E - \tau}{2} - \frac{D}{h} \right)\]
that becomes
\[  -E - \frac{D}{h}B\left(\! \frac{Eh}{D}\!\right)\]
and that can be rewritten as
\[ - \frac{D}{h} \left[ \frac{Eh}{D} + B\left(\! \frac{Eh}{D} \! \right)\right]\]
applying the lemma~\ref{bernulli_inverse} we obtain the opposite of the coefficient of $u_{k + 1}$ 
in the equation~\ref{eq:sg_system}.

Finally, the right hand side of both the equations does not depend on $\tau$ when the elements have 
the same length. Therefore, it is easy to check that they are opposite of each other.
\end{proof}

\begin{prop}
The value of $\tau_0$ is an even function respect to $E$.
\end{prop}
\begin{proof}
 The proof follows directly from lemma~\ref{bernulli_even}.
\end{proof}





\end{document}
